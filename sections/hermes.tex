The Heliophysics Environmental and Radiation Measurement Experiment Suite (HERMES) is a NASA mission to characterize the space environment in the interplanetary medium and the terrestrial magnetotail.
HERMES includes an ion mass spectrometer, an electron electrostatic analyzer, a proton and electron telescope for energetic particles, and a set of magnetometers.
It will fly as a hosted external payload on the Lunar Gateway space station whose purpose is to serve as a communication hub, short-term habitation module, and laboratory for astronauts as part of the Artemis program.
The first two modules of the the Lunar Gateway consists of the Power and Propulsion Element (PPE) and the Habitation and Logistics Outpost (HALO).
The PPE is a a high-power, solar electric propulsion spacecraft that will serve as the command and communications center of the Gateway and supply the station with electrical power while the HALO will host HERMES as an external payload and provides multidirectional docking ports, power distribution, command and control systems, and a functional pressurized volume with environmental control and life support systems to support a crew of four for at least 30 days.
The two modules will launch together no earlier than 2027.
After an approximately one year transit to the Moon, the lunar Gateway will enter a near-rectilinear halo orbit (NRHO) around the Moon at which point HERMES will begin its science campaign.
Additional components of the Gateway are expected to be delivered and integrated while in lunar orbit.
HERMES is meant to be a pathfinder for future science payloads on human-exploration vessels, for which there is a pragmatic interest in the variable radiation environment.
Thus, with observations enabled by the ARTEMIS Program, HERMES is expected to be enabling of future exploration missions.
Additionally, although HERMES objectives have a space-weather focus, the measurements also can be useful for studies of the Moon.

The HERMES Science Goals and Objectives are

\begin{table}[h]
\caption{HERMES Science Goals and Objectives}\label{table_hermes_stm}%
\begin{tabular}{@{}ll@{}}
\toprule
Goals & Objective 2 \\
\midrule
Determine mechanisms of solar wind mass and energy transport    & data 1    \\
Characterize energy, topology, and ion composition in the deep magnetotail    & data 4    \\
Establish observational capabilities of an on-board pathfinder payload measuring local space weather to support deep-space and long-term human exploration    & data 7    \\
\botrule
\end{tabular}
\end{table}

The HERMES instrument suite consists of the following instruments

\begin{table}[h]
\caption{HERMES Instruments}\label{table_hermes_instr}%
\begin{tabular}{@{}llll@{}}
\toprule
Name & Measurement & Performance & Provider \\
\midrule
Electron Electrostatic Analyser (EEA) & Electron Velocity Distribution & & NASA GSFC (Dr. Daniel Gershman) \\
Miniaturized Electron pRoton Telescope (MERiT) & Proton and Electron Flux & & NASA GSFC (Dr. Shri Kanekal) \\
Noise Eliminating Magnetometer In a Small Integrated System (NEMISIS) & Magnetic Field Vector & & NASA GSFC Dr. Eftyhia Zesta &
Solar Probe Analyzer - Ions (SPAN-I) & Ion Velocity Distribution (M/Q resolved) & & UC Berkeley (Dr. Roberto Livi) \\

\botrule
\end{tabular}
\end{table}

\subsection{Data Products}



\begin{itemize}
    \item Science background
    \item Mission Overview
    \item Science Objectives
\end{itemize}
